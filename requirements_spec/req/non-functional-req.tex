%Overview
\section{Non-Functional Requirements}

The \underline{non-functional requirements} describe how the functional requirements are met by specifying a set of criteria and characteristics that can be used to provide a quantitative, comprehensive assessment of the system. Therefore, non-functional requirements can be divided into two mutually exclusive groups of requirements: Constraints and quality attribute requirements.\par

\underline{Constraints} are limitations imposed by the functional requirements and that add direct limitations on the types of technology developers can use in the system design or implementation phase. Constraints are displayed in the same way as use cases are.\par

Quality attribute requirements (or simply \underline{quality atttributes}) is a set of characteristics of one word expressed and elaborated through a set of quality scenarios in which the reader is appointed how the achievement of these characteristics to be measured. These scenarios written in the form of a table expresses how the system should respond to certain defined events that expose or specific forms of user interaction. Quality attributes describe the system’s intended behavior within the environment for which it was built. They provide the means for measuring the fitness and suitability of a product.\par

\subsection{Constraints}

%TODO: Input table of non-functional requirements, display first those elicited by the customers directly.

\subsection{Quality Attributes}

Our customers have explicitly stated that the system must be both usable and secure. Of course, they left it to us to figure out how it will contain said quality properties. We have opted to illustrate the quality requirements in the form of scenarios. From the functional requirements we have identified the following major quality attributes, which drive our choice of patterns and architectural views: Usability, security, modifiability, and testability (we have appended the latter attribute as we are intending to setup and deploy a testing framework). The scenarios surrounding each quality attribute shall serve as pointers as to what tactics and design patterns we must impolement to fulfill them. That is the purpose of these concrete scenarios.

%TODO: QAS. seperate command implemented in setup.
