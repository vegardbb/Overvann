\section{Functional requirements}

% TODO: Write a short bit on how the team defines a functional requirement, and what the tables mean. One functional requirement may spawn several non-functional requirements, thus each nfr has a reference field back to the FR that nescessitated them. Nonfnctional requirements are either Quality Requirements or Constraints.

% beginfr and stopreq are functions defined in setup. beginfr takes one argument, the title of the requirement listing. fritem takes in an id, a description, a priority rank {Low, Medium, High} indicating how desirable it is to fulfill the requirement during the project, and a complexity rank {Easy, Medium, Hard} indicating how difficult the requirment is estimated to fulfill.
\beginfr{Listing of Functional Requirements}
	\fritem{ID}{C-ID}{Description}{Low}{Low}
	\fritem{FR-01}{C-01}{The front page has a link to a listing page for projects}{High}{Medium}
\stopreq

% TODO: Separate tables for each module to be implemented
